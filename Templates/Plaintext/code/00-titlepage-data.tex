% @(#) 00-titlepage-data.tex: theDedicatedFolderName
% Last-edited: 2022.06.12.2225.Sun -- Danny Quah (me@DannyQuah.com)
% ----------------------------------------------------------------
% Revision History:
%  % 2022.06.10.2223.Fri -- Danny Quah (me@DannyQuah.com) 
%    First draft - from Templates/Plaintext/00-titlepage.tex
% $
% $Log$
% ----------------------------------------------------------------
%
\newcommand\theTitle{%
Mobility and Political Upheaval\\
in an Age of Inequality%
}
\newcommand\theShortTitle{%
Mobility and Political Upheaval%
}
\newcommand\theAuthor{%
Danny Quah%
}
\newcommand\theAuthorSurname{%
Quah%
}
\newcommand\theAffiliation{%
Lee Kuan Yew School of Public Policy, NUS%
}
\newcommand\theAddress{%
469C Bukit Timah Road, Singapore 259772%
}
\newcommand\theEmail{%
D.Quah@nus.edu.sg%
}
\newcommand\theDate{%
%  ${ } $ \\
\dqMonthYear{}
%% \dqToday\
% (Draft manuscript incomplete)
}
\newtoggle{bAbstract} \togglefalse{bAbstract}
\toggletrue{bAbstract}
\newcommand\theAbstract{%
\noindent
Appropriate public policy on inequality
hinges critically on
understanding inequality's effects on
the living conditions of the poor,
on social mobility,
and on nationalist populism.
This paper describes two empirical regularities.
First, an
increase in inequality typically does not coincide with
immiserisation of the poor and lower middle class.
Over 80\% of economies where inequality has risen since 2000 have
also increased the average incomes of their populations' bottom 50\%.
%% Beyond this, data show remarkably little relation between
%% mobility, on the one hand, and inequality and its dynamics, on the other.
Second,
%% political upheaval can emerge around a
%% narrative of
%% inequality and immobility even when
%% in reality the situation of the poor improves and
%% society is not becoming more unequal.
for political upheaval,
individual well-being and expectations on its trajectory
matter more than inequality.
When these causal factors diverge,
the role of inequality is, thus, diminished.
Public policy needs to counter
misinterpretation and misinformation on inequality
with rigorous analysis and empirical evidence.
%% This is especially important in discussions on inequality and social mobility,
%% where political and populist rhetoric can easily dominate national conversations.
% \medskip
}
%
\newtoggle{bJELcodes} \togglefalse{bJELcodes}
\toggletrue{bJLEcodes}
\newcommand\theJELcodes{% https://www.aeaweb.org/econlit/jelCodes.php?view=jel
 %C10,% Econometric and Statistical Methods - General
 D21, % Firm Behaviour: Theory
 D22, % Firm Behaviour: Empirical Analysis
 D24, % Production, Cost, Capital, Productivity, Capacity
 D25, % Intertemporal Firm Choice: Investment, Capacity, Training
 D31, % Personal Income, Wealth, and Their Distributions
 D63, % Equity, Justice, Inequality, and other Normative Criteria, Measurement
 %F51, % International Conflicts
 F52, % Economic Nationalism
 O40, % Economic Growth - General
 O57%,% Comparative Studies of Countries
 %Y10,% Data: Tables and Charts
 \medskip
}
%
\newtoggle{bKeyWords} \togglefalse{bKeyWords}
\toggletrue{bKeyWords}
\newcommand\theKeywords{%
bottom~50\%;
disinformation;
growth;
income gap;
income inequality;
top~10,\%;
upward mobility%
\medskip
}
%
\newtoggle{bTheThanks} \togglefalse{bTheThanks}
\toggletrue{bTheThanks}
\newcommand\theThanks{%
%% Preliminary.
I thank ...
The author is also Faculty Associate, Centre on Asia and Globalisation
at the Lee Kuan Yew School of Public Policy.
}
%
%% If published or forthcoming
\newcommand\thePublMonth{n.a.}
\newcommand\thePublYear{n.a.}
\newcommand\thePublVol{n.a.}
\newcommand\thePublIssue{n.a.}
%
% ^L
% Local Variables:
% mode: TeX
% end:
% eof 00-titlepage-data.tex
